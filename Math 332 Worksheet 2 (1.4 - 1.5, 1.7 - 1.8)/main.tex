\documentclass[letter,11pt]{article}
\usepackage{xcolor}
\usepackage{float}
\usepackage{amsthm}
\usepackage{amssymb}
\usepackage{wrapfig}
\usepackage{tabularx}
\usepackage{titlesec}
\usepackage{tikz}
\usepackage{geometry}
\usepackage{verbatim}
\usepackage{enumitem}
\usepackage{fancyhdr}
\usepackage{multicol}
\usepackage{systeme}
\usepackage{booktabs}
\usepackage{graphicx}
\usepackage{mathtools}
\usepackage{booktabs}
\usepackage{svg}
\usepackage[most]{tcolorbox}
\usepackage[T1]{fontenc}
\usetikzlibrary{trees}
\setlength{\multicolsep}{0pt} 
\pagestyle{fancy}
%\fancyhf{} % clear all header and footer fields
\fancyhead{}\fancyfoot{}
\fancyhead[R]{\textbf{\thepage}}
\fancyhead[L]{Aiden M. Rosenberg, MMXXIV A.D. }
\addtolength{\headwidth}{3cm}
\addtolength{\headheight}{1cm}

\usepackage{pgfplots}
\pgfplotsset{compat=1.17}
\usepgfplotslibrary{fillbetween}

\usepackage{pst-plot}
\usepgfplotslibrary{polar}

\renewcommand{\headrulewidth}{1pt}
\renewcommand{\footrulewidth}{0pt}
\geometry{left=1.5cm, top=2.5cm, right=1.5cm, bottom=2cm}

\usepackage[most]{tcolorbox}

\usepackage{tasks}
\settasks{
	label=(\Alph*.),
	label-width=21pt
}

\raggedright
\setlength{\tabcolsep}{0in}

% Sections formatting
\titleformat{\section}{
  \vspace{-4pt}\scshape\raggedright\large
}{}{0em}{}[\color{black}\titlerule \vspace{-7pt}]

\titleformat{\subsection}[block]
  { \vspace{4pt}\bfseries\centering}
  {}{0em}{}

\newcommand{\pvec}[1]{\vec{#1}\mkern2mu\vphantom{#1}}

\begin{document}

\thispagestyle{empty}

%----------HEADING-----------------

\parbox{2.35cm}{%
	\includesvg[width=2.3cm]{logo.svg}
}
\parbox{0.3cm}{\hspace{0.3cm}}
\parbox{\dimexpr\linewidth-5cm\relax}{
	\setlength{\tabcolsep}{0.5em}
	\def\arraystretch{1.25}
	\begin{tabular}{@{}llll@{}}
		\toprule
		\multicolumn{4}{c}
		{\hspace{-0.5em}\textbf{Assignment}: Worksheet 2 (\S 1.4 - \S1.5, \S1.7 - \S1.8)} \\ \midrule
		\textbf{Name:}   & Aiden M. Rosenberg  & \textbf{Professor:} & Dr. Terry Bridgman Ph.D \\
		\textbf{Course:} & Linear Algebra          & \textbf{Date:}      & \today \: A.D.   \\ \bottomrule
	\end{tabular}}
\parbox{0.3cm}{\hspace{0.3cm}}
\vspace{1cm}

\section{Problem 1}
Determine, by inspection (i.e., you do not need to do any row reduction for these problems to determine the answer), if the given vectors form a linearly dependent or linearly independent set. Provide a \textit{brief} justification for your answer.
\begin{enumerate}[label = \roman*.]
    \item $\vec{\boldsymbol{u}}_{1}=\begin{bmatrix}14 \\ -6\end{bmatrix}, \vec{\boldsymbol{u}}_{2}=\begin{bmatrix}-7 \\ 3\end{bmatrix}$
    \item $\vec{\boldsymbol{u}}_{1}=\begin{bmatrix}3 \\ -1\end{bmatrix}, \vec{\boldsymbol{u}}_{2}=\begin{bmatrix}6 \\ -5\end{bmatrix}, \vec{\boldsymbol{u}}_{3}=\begin{bmatrix}1 \\ 4\end{bmatrix}$
    \item $\vec{\boldsymbol{u}}_{1}=\begin{bmatrix}1 \\ -8 \\ 3\end{bmatrix}, \vec{\boldsymbol{u}}_{2}=\begin{bmatrix}0 \\ 0 \\ 0\end{bmatrix}, \vec{\boldsymbol{u}}_{3}=\begin{bmatrix}-7 \\ 1 \\ 12\end{bmatrix}$
    \item $\vec{\boldsymbol{u}}_{1}=\begin{bmatrix}6 \\ -4 \\ 2\end{bmatrix}, \vec{\boldsymbol{u}}_{2}=\begin{bmatrix}3 \\ -2 \\ -1\end{bmatrix}$ 
\end{enumerate}

\begin{tcolorbox}[boxrule=1mm, width=(.9\linewidth),before=\hfill,after=\hfill,adjusted title={Problem \#1 Solutions}]
\begin{enumerate}[label = \roman*.]
    \item The given vectors are \underline{linearly dependant} since the matrix equation $$a_{1}\begin{bmatrix}14 \\ -6\end{bmatrix}+ a_{2}\begin{bmatrix}-7 \\ 3\end{bmatrix}=\begin{bmatrix} 0\\0\end{bmatrix},$$ has a solution of $a_1=-\frac{1}{2}$ which is in addition to the trivial solution.
    \item Since the set contains more vectors than there are entries in each vector, therefore the set is \underline{linearly dependant}. That is, any set  $\{\vec{\boldsymbol{v}}_{1}, \ldots, \vec{\boldsymbol{v}}_{p}\}$ in $\mathbb{R}^{n}$ is linearly dependent if $p > n$. In this case there are three vectors in the set $p = 3$ and there are only two entries in each vector $n=2$
    \item If, a set $\{\vec{\boldsymbol{v}}_{1}, \ldots, \vec{\boldsymbol{v}}_{p}\}$ in $\mathbb{R}^2$ contains the zero vector, then the set is linearly. The set is \underline{linearly dependant} since $\vec{\boldsymbol{u}}_{3}$ has no entry i.e. zero vector.
    \item The given vectors are \underline{linearly independent} since the matrix equation $$a_{1}\begin{bmatrix}6 \\ -4\\2\end{bmatrix}+ a_{2}\begin{bmatrix}3 \\ -2\\-1\end{bmatrix}=\begin{bmatrix} 0\\0\\0\end{bmatrix},$$ has only the trivial solution.
    
\end{enumerate}

\end{tcolorbox}

\section{Problem 2}
Consider the following sets of vectors in $\mathbb{R}^{3}$
\begin{tasks}(4)
    \task $\left\{\vec{\boldsymbol{u}}_{1}\right\}$
    \task $\left\{\vec{\boldsymbol{u}}_{1}, \vec{\boldsymbol{u}}_{2}\right\}$
    \task $\left\{\vec{\boldsymbol{u}}_{1}, \vec{\boldsymbol{u}}_{2}, \vec{\boldsymbol{u}}_{3}\right\}$
    \task $\left\{\vec{\boldsymbol{u}}_{1}, \vec{\boldsymbol{u}}_{2}, \vec{\boldsymbol{u}}_{3}, \vec{\boldsymbol{u}}_{4}\right\}$
\end{tasks}

\begin{enumerate}[label = \roman*.]
    \item Which of these sets could possibly be linearly independent? Justify your answer.
    \item Which of these sets could possibly be linearly independent and span $\mathbb{R}^{3}$? Justify your answer
\end{enumerate}

\begin{tcolorbox}[boxrule=1mm, width=(.9\linewidth),before=\hfill,after=\hfill,adjusted title={Problem \# 2 Solutions}]
    \begin{enumerate}[label = \roman*.]
        \item  The sets \( A \), \( B \), and \( C \) could each be linearly independent.  For set \( A \), it is linearly independent if and only if \( \vec{u}_1 \) is nonzero. Set \( B \) could be linearly independent if \( \vec{\boldsymbol{u}}_1 \) and \( \vec{a}_2 \) are not multiples of each other. The set \( C \) could be linearly independent if none of the vectors in the set \( S = \{ \vec{\boldsymbol{u}}_1, \vec{\boldsymbol{u}}_2, \vec{\boldsymbol{u}}_3 \} \) can be expressed as a linear combination of the others. However, set \( D \) cannot be linearly independent because \( \vec{a}_4 \) can be represented as a linear combination of the vectors in \( \{ \vec{\boldsymbol{u}}_1, \vec{\boldsymbol{u}}_2, \vec{\boldsymbol{u}}_3 \} \), considering the defined set \( \mathbb{R}^3 \), thereby making the set liner dependent.
        \item Set \( C \) could span \( \mathbb{R}^3 \) if each of the vectors in \( S \) (defined in the previous response) is nonzero. This is true because all vectors in the space \( \mathbb{R}^3 \) could be generated from scalar multiples of \( S \). In other words, if $\vec{\boldsymbol{b}}$ is some vector in $\mathbb{R}^3$, then $\vec{\boldsymbol{b}} = x_{1} \vec{\boldsymbol{u}}_1+x_{2} \vec{\boldsymbol{u}}_2+ x_{2} \vec{\boldsymbol{u}}_3$, has a solution. 
    \end{enumerate}
\end{tcolorbox}
\newpage
\section{Problem 3}
Determine if the following vector set is linearly independent. Explain how you came to your conclusion.

$$ \vec{\mathbf{u}}_{1}=\begin{bmatrix} -1 \\ 4 \\ -2 \\ -3 \end{bmatrix}, \quad \vec{\mathbf{u}}_{2}=\begin{bmatrix} 3 \\ -13 \\ 7 \\ 7 \end{bmatrix}, \quad \vec{\mathbf{u}}_{3}=\begin{bmatrix} -2 \\ 1 \\ 9 \\ -5 \end{bmatrix} $$

\begin{tcolorbox}[boxrule=1mm, width=(.9\linewidth),before=\hfill,after=\hfill,adjusted title={Problem \# 3 Solutions}]
After performing elementary row operations to row-reduce an arbitrary matrix $A_{m \times n}$, the emergence of a row of zeros indicates that the rows are dependent. Row reduction preserves the row space, i.e., the space spanned by the rows of the matrix. Therefore, encountering a row of zeros during the reduction process signifies that the row space is spanned by fewer than $m$ elements. Consequently, the original rows must have been dependent. Reducing the matrix created from the vector set above (see reduction below), it can be seen that the vector set is \underline{linearly dependent} since it contains a free variable in the fourth row of the reduced matrix below.

\tcblower 
$$\begin{bmatrix}
-1 & 3 & -2 \\
4 & -13 & 1\\
-2 & 7 & 9\\
-3 & 7 & 5
\end{bmatrix} \xrightarrow{R_1 = -R_1}
\begin{bmatrix}
1 & -3 & 2 \\
4 & -13 & 1\\
-2 & 7 & 9\\
-3 & 7 & 5
\end{bmatrix}\xrightarrow{R_2 = R_2 -4R_1}
\begin{bmatrix}
1 & -3 & 2 \\
0 & -1 & -7\\
-2 & 7 & 9\\
-3 & 7 & 5
\end{bmatrix}\xrightarrow{R_3 = R_3 +2R_1}$$

$$\begin{bmatrix}
1 & -3 & 2 \\
0 & -1 & -7\\
0 & 1 & 13\\
-3 & 7 & 5
\end{bmatrix} \xrightarrow{R_4=R_4+3R_1}
\begin{bmatrix}
1 & -3 & 2 \\
0 & -1 & -7\\
0 & 1 & 13\\
0 & -2 & 11
\end{bmatrix}\xrightarrow{R_2 = -R_2}
\begin{bmatrix}
1 & -3 & 2 \\
0 & 1 & 7\\
0 & 1 & 13\\
0 & -2 & 11
\end{bmatrix}\xrightarrow{R_1 = R_1 +3 R_2}$$

$$\begin{bmatrix}
1 & 0 & 23 \\
0 & 1 & 7\\
0 & 1 & 13\\
0 & -2 & 11
\end{bmatrix} \xrightarrow{R_3=R_3-R_2}
\begin{bmatrix}
1 & 0 & 23 \\
0 & 1 & 7\\
0 & 0 & 6\\
0 & -2 & 11
\end{bmatrix}\xrightarrow{R_4=R_4+2R_2}
\begin{bmatrix}
1 & 0 & 23 \\
0 & 1 & 7\\
0 & 0 & 6\\
0 & 0 & 25
\end{bmatrix}\xrightarrow{R_3 = \frac{1}{6}R_3}$$

$$\begin{bmatrix}
1 & 0 & 23 \\
0 & 1 & 7\\
0 & 0 & 1\\
0 & 0 & 25
\end{bmatrix} \xrightarrow{R_1=R_1-23R_2}
\begin{bmatrix}
1 & 0 & 0 \\
0 & 1 & 7\\
0 & 0 & 1\\
0 & 0 & 25
\end{bmatrix}\xrightarrow{R_4=R_4-25R_3}
\begin{bmatrix}
1 & 0 & 0 \\
0 & 1 & 0\\
0 & 0 & 1\\
0 & 0 & 0
\end{bmatrix}$$

\end{tcolorbox}
\newpage
\section{Problem 4}
The myology clinic at a university research hospital helps patients recover muscle mass lost due to illness. The clinic uses exercise training and nutritional powder to meet the patient's needs. The nutritional powders are created using some or all of 4 brands that the clinic keeps in stock. For example,

\begin{table}[h]
\centering
\begin{tabular}{@{}l@{\hspace{1em}}l@{\hspace{1em}}l@{\hspace{1em}}l@{\hspace{1em}}l@{}}
\multicolumn{5}{c}{Brand}         \\ \toprule
              & A   & B  & C & D  \\ \midrule
Protein       & 16 & 22 & 18 & 18 \\
Fat           & 2  & 4  & 0  & 2  \\
Carbohydrates \: & 8  & 4  & 4  & 6  \\ \bottomrule
\end{tabular}
\end{table}
represents the nutritional components for brands A, B, C and D, from which the hospitals constructs the powder for any given patient's needs. However, it is expensive to stock all 4 brands. Determine if the hospital can reduce their costs by eliminating a brand that would still allow them to produce any combination of protein, fat, and carbohydrates they'd like. If so, use the corresponding vector equation to also give the corresponding mixtures of the remaining brands that can be used to replace the redundant brands.
\begin{tcolorbox}[boxrule=1mm, width=(.9\linewidth),before=\hfill,after=\hfill,adjusted title={Problem \# 4 Solutions}]
The hospital can remove brand D from its options while still being able to create any combination of protein, fat, and carbohydrates. This is indicated by the presence of a pivot in each row of the reduced row echelon form of the matrix below. Essentially, brand D can be expressed as a linear combination of brands A, B, and C. This relationship can be represented by the matrix equation:$$\vec{\boldsymbol{d}} = x_{1} \begin{bmatrix} 1\\ 0\\ 0\end{bmatrix} + x_{2}\begin{bmatrix} 0\\ 1\\ 0\end{bmatrix} + x_{3}\begin{bmatrix} 0\\ 0\\ 1\end{bmatrix} = \begin{bmatrix} 0.5\\ 0.25\\ 0.25\end{bmatrix},$$ 
where $x_{1} = 0.5$, $x_{2} = 0.25$ and $x_{3} = 0.25$. This means that to get the nutrient composition represented by brand D, you can take 50\% of the composition of brand A, 25\% of brand B, and 25\% of brand C.

\tcblower 
$$\begin{bmatrix}
16 & 22 & 18 & 18 \\
2 & 4 & 0 & 2\\
8 & 4 & 4 & 6
\end{bmatrix} \xrightarrow{R_1 = \frac{R_1}{16}}
\begin{bmatrix}
1 & \frac{11}{8} & \frac{9}{8} & \frac{9}{8} \\
2 & 4 & 0 & 2\\
8 & 4 & 4 & 6
\end{bmatrix}\xrightarrow{R_2 = R_2 -2R_1}
\begin{bmatrix}
1 & \frac{11}{8} & \frac{9}{8} & \frac{9}{8} \\
0 & \frac{5}{4} & -\frac{9}{4} & -\frac{1}{4}\\
8 & 4 & 4 & 6
\end{bmatrix}\xrightarrow{R_3 = R_3 -8R_1}$$

$$\begin{bmatrix}
1 & \frac{11}{8} & \frac{9}{8} & \frac{9}{8} \\
0 & \frac{5}{4} & -\frac{9}{4} & -\frac{1}{4}\\
0 & -7 & -5 & -3
\end{bmatrix} \xrightarrow{R_{2}=\frac{4}{5}R_2}
\begin{bmatrix}
1 & \frac{11}{8} & \frac{9}{8} & \frac{9}{8} \\
0 & 1 & -\frac{9}{4} & -\frac{1}{5}\\
0 & -7 & -5 & -3
\end{bmatrix}\xrightarrow{R_1=R_1-\frac{11}{8}R_2}
\begin{bmatrix}
1 & 0 & \frac{18}{5} & \frac{7}{5} \\
0 & 1 & -\frac{9}{5} & -\frac{1}{5}\\
0 & -7 & -5 & -3
\end{bmatrix}\xrightarrow{R_3=R_3+7R_2}$$

$$\begin{bmatrix}
1 & 0 & \frac{18}{5} & \frac{7}{5} \\
0 & 1 & -\frac{9}{5} & -\frac{1}{5}\\
0 & 0 & -\frac{88}{5} & \frac{-22}{5}
\end{bmatrix} \xrightarrow{R_3=-\frac{5}{88}R_3}
\begin{bmatrix}
1 & 0 & \frac{18}{5} & \frac{7}{5} \\
0 & 1 & -\frac{9}{5} & -\frac{1}{5}\\
0 & 0 & 1 & \frac{1}{4}
\end{bmatrix}\xrightarrow{R_1=R_1-\frac{18}{5}R_3}
\begin{bmatrix}
1 & 0 & 0 & \frac{1}{2} \\
0 & 1 & -\frac{9}{5} & -\frac{1}{5}\\
0 & 0 & 1 & \frac{1}{4}
\end{bmatrix}\xrightarrow{R_2 = \frac{9}{5}R_3}$$

$$\begin{bmatrix}
1 & 0 & 0 & \frac{1}{2} \\
0 & 1 & 0 & \frac{1}{4}\\
0 & 0 & 1 & \frac{1}{4}
\end{bmatrix}$$
\end{tcolorbox}
\newpage
\section{Problem 5}
Suppose that

$$\vec{\boldsymbol{a}}_{1}=\begin{bmatrix} 1 \\ 7 \\ -2 \end{bmatrix}, \vec{\boldsymbol{a}}_{2}=\begin{bmatrix}3\\0 \\ 1 \end{bmatrix} \text {, and } \vec{\boldsymbol{a}}_{3}=\begin{bmatrix} 5 \\ 2 \\ -6 \end{bmatrix} \text { with } A=\begin{bmatrix}\vec{\boldsymbol{a}}_{1} & \vec{\boldsymbol{a}}_{2} & \vec{\boldsymbol{a}}_{3}\end{bmatrix} $$

\begin{enumerate}[label = \roman*.]
    \item Are the columns of $A$ are linearly independent? Why or why not?
    \item Do the columns of $A$ span $\mathbb{R}^{3}$ ? Why or why not?
    \item What can you say about the solutions to $A \vec{\boldsymbol{x}}=\vec{\boldsymbol{b}}$ ? Justify your answer.
\end{enumerate}
\begin{tcolorbox}[boxrule=1mm, width=(.9\linewidth),before=\hfill,after=\hfill,adjusted title={Problem \# 5 Solutions}]
\begin{enumerate}[label = \roman*.]
    \item After row reducing matrix $A$, it's evident that $A$ is linearly independent because there exists a pivot in each column of the reduced row echelon form of $A$ (see below). This occurs because in each pivot column, all entries below the pivot are zero. Consequently, a linear combination of these columns cannot yield the zero vector. This fact is apparent in the matrix equation: $$ x_{1} \begin{bmatrix} 1\\ 0\\ 0\end{bmatrix} + x_{2}\begin{bmatrix} 0\\ 1\\ 0\end{bmatrix} + x_{3}\begin{bmatrix} 0\\ 0\\ 1\end{bmatrix}=\vec{\boldsymbol{d}},$$ where the only solution is the trivial one.
    \item  After performing row reduction on matrix $A$, it becomes evident that the columns of $A$ span $\mathbb{R}^{3}$. This determination stems from observing that there is a pivot in every row of the reduced row echelon form of $A$, as illustrated below: $$ \begin{bmatrix} 1 & 0 & 0 \\ 0 & 1 & 0 \\ 0 & 0 & 1 \\ \end{bmatrix} $$ In other words, for each $\vec{\boldsymbol{b}}$ in $\mathbb{R}^m$, the equation $A \vec{\boldsymbol{x}} = \vec{\boldsymbol{b}}$ has a solution. This signifies that each $\vec{\boldsymbol{b}}$ in $\mathbb{R}^m$ is a linear combination of the columns of $A$.
    \item The solutions to the linear system $A \vec{\boldsymbol{x}} = \vec{\boldsymbol{b}}$ are both consistent and unique. This is due to the fact that, as stated in part (i), the zero vector is a solution, satisfying the equation $A \vec{\boldsymbol{0}} = \vec{\boldsymbol{b}}$. Additionally, there are no free variables in the row-reduced matrix, implying that each variable is uniquely determined by the system of equations. The absence of free variables ensures that there is only one possible solution, leading to uniqueness. Therefore, the solution space consists solely of the zero vector, confirming both consistency and uniqueness for the linear system.
\end{enumerate}

\tcblower 
$$\begin{bmatrix}
1 & 3 & 5 \\
7 & 0 &  2\\
-2 & 1 & -6
\end{bmatrix} \xrightarrow{R_2 = R_2-7R1}
\begin{bmatrix}
1 & 3 & 5 \\
0 & -21 &  -33\\
-2 & 1 & -6
\end{bmatrix}\xrightarrow{R_3 = R_3 +2R_1}
\begin{bmatrix}
1 & 3 & 5 \\
0 & -12 &  -33\\
0 & 7 & 4
\end{bmatrix}\xrightarrow{R_{2}=-\frac{1}{21}R_2}$$

$$\begin{bmatrix}
1 & 3 & 5 \\
0 & 1 &  \frac{11}{7}\\
0 & 7 & 4
\end{bmatrix} \xrightarrow{R_1=R_1-3R_2}
\begin{bmatrix}
1 & 0 & \frac{2}{7} \\
0 & 1 &  \frac{11}{7}\\
0 & 7 & 4
\end{bmatrix}\xrightarrow{R_3=R_3-7R_2}
\begin{bmatrix}
1 & 0 & \frac{2}{7} \\
0 & 1 &  \frac{11}{7}\\
0 & 0 & -7
\end{bmatrix}\xrightarrow{R_3=-\frac{1}{7}R_3}$$

$$\begin{bmatrix}
1 & 0 & \frac{2}{7} \\
0 & 1 &  \frac{11}{7}\\
0 & 0 & 1
\end{bmatrix} \xrightarrow{R_1=R_1-\frac{2}{7}R_3}
\begin{bmatrix}
1 & 0 & 0 \\
0 & 1 &  \frac{11}{7}\\
0 & 0 & 1
\end{bmatrix}\xrightarrow{R_2=R_2-\frac{11}{7}R_3}
\begin{bmatrix}
1 & 0 & 0 \\
0 & 1 &  0\\
0 & 0 & 1
\end{bmatrix}$$

\end{tcolorbox}

\newpage
\section{Problem 6}
Given the following transformations $T$, determine whether $T$ a linear transformation. If it is, show that $T$ satisfies both properties of a linear transformation. If not, show that $T$ fails to satisfy at least one of the properties.
\begin{tcolorbox}
     A transformation (or mapping) $T$ is linear if:
    \begin{enumerate}
        \item $T\left(\boldsymbol{u}+\boldsymbol{v}\right) = T\left(\boldsymbol{u}\right)+T\left(\boldsymbol{v}\right)$ for all $\boldsymbol{u}$, $\boldsymbol{v}$ in the domain of $T$;
        \item $T\left(r\boldsymbol{u}\right) = r T\left(\boldsymbol{u}\right)$
    \end{enumerate}
\end{tcolorbox}
\begin{enumerate}[label = \roman*.]
    \item Let $\vec{v}=\begin{bmatrix}x \\ y\end{bmatrix}$. $T(\vec{v})=\begin{bmatrix}4 x-2 y \\ 3 x \\ 5 y\end{bmatrix}$.
    \item Let $\vec{v}=\begin{bmatrix}x \\ y\end{bmatrix}$. $T(\vec{v})=\begin{bmatrix}2 y \\ x+4\end{bmatrix}$.
\end{enumerate}

\begin{tcolorbox}[boxrule=1mm, width=(.9\linewidth),before=\hfill,after=\hfill,adjusted title={Problem \# 6.i Solutions}]
$$T\left(\begin{bmatrix}x \\ y\end{bmatrix}\right) = \begin{bmatrix} 4x-2y\\ 3x+0y\\0x+5y \end{bmatrix} = \begin{bmatrix} 4 & -2\\ 3 & 0\\ 0 & 5\end{bmatrix}\begin{bmatrix}x\\y \end{bmatrix}$$
        Let $\boldsymbol{v} = \begin{bmatrix} a\\b \end{bmatrix}\quad \text{and} \quad \boldsymbol{u} = \begin{bmatrix} c\\d \end{bmatrix}\Longrightarrow \boldsymbol{v}+\boldsymbol{u} = \begin{bmatrix} c+a\\ d+b \end{bmatrix}$  
\tcblower
    \textbf{\textit{Proof 1}}
    
    $$T\left(\boldsymbol{v}+\boldsymbol{u}\right) = \begin{bmatrix} 4(c+a) - 2(d+b)\\ 3(c+a)+0(d+b)\\ 0(c+a)+5(d+b) \end{bmatrix}$$

    $$T\left(\boldsymbol{v}\right)+T\left(\boldsymbol{u}\right) = \begin{bmatrix}4c-2d\\ 3c+0d\\0c+5d\end{bmatrix} + \begin{bmatrix} 4a-2b\\ 3a+0b\\0a+5b \end{bmatrix} = \begin{bmatrix}4(c+a)-2(d+b)\\ 3(c+a)+0(d+b)\\ 0(c+a)+ 5(d+b)\end{bmatrix} $$
    
    $$T\left(\boldsymbol{v}+\boldsymbol{u}\right)=T\left(\boldsymbol{v}\right)+T\left(\boldsymbol{u}\right)$$ \qed
    
    \textbf{\textit{Proof 2}}
    
    $$T\left(r\boldsymbol{u}\right) =  \begin{bmatrix} 4(rc)-2(rd)\\ 3(rc)+0(rd)\\ 0(rc)+5(rd) \end{bmatrix} = r \begin{bmatrix} 4c-2d\\3c+0d\\0c+5d \end{bmatrix}$$

    $$rT(\boldsymbol{u}) = r \begin{bmatrix} 4c-2d\\3c+0d\\0c+5d \end{bmatrix}$$

     $$T\left(r\boldsymbol{u}\right) = rT(\boldsymbol{u})$$ \qed
\end{tcolorbox}

\begin{tcolorbox}[boxrule=1mm, width=(.9\linewidth),before=\hfill,after=\hfill,adjusted title={Problem \# 6.ii Solutions}]
$$T\left(\begin{bmatrix}x \\ y\end{bmatrix}\right) = \begin{bmatrix} 4x-2y\\ 3x+0y\\0x+5y \end{bmatrix} $$
        Let $\boldsymbol{v} = \begin{bmatrix} a\\b \end{bmatrix}\quad \text{and} \quad \boldsymbol{u} = \begin{bmatrix} c\\d \end{bmatrix}\Longrightarrow \boldsymbol{v}+\boldsymbol{u} = \begin{bmatrix} c+a\\ d+b \end{bmatrix}$  
\tcblower
    \textbf{\textit{Proof 1}}
    
    $$T\left(\boldsymbol{v}+\boldsymbol{u}\right) = \begin{bmatrix} 0(c+a) + 2(d+b)\\ (c+a)+4+0(d+b) \end{bmatrix}$$

    $$T\left(\boldsymbol{v}\right)+T\left(\boldsymbol{u}\right) = \begin{bmatrix}0(a)+2(b)\\ (a+4)+0(b)\end{bmatrix} + \begin{bmatrix} 0(c)+2(c)\\ (c+4)+0(d) \end{bmatrix} = \begin{bmatrix}0(c+a)+2(b+b)\\ (c+a)+4+0(b+d)\end{bmatrix} $$
    
    $$T\left(\boldsymbol{v}+\boldsymbol{u}\right) \neq T\left(\boldsymbol{v}\right)+T\left(\boldsymbol{u}\right)$$ 
    
    \textbf{\textit{Proof 2}}
    
    $$T\left(r\boldsymbol{u}\right) =  \begin{bmatrix}0(ra)+2(rb)\\ (ra+4)+0(rb)\end{bmatrix} $$

    $$rT(\boldsymbol{u}) = r  \begin{bmatrix}0(a)+2(b)\\ (a+4)+0(b)\end{bmatrix}$$

     $$T\left(r\boldsymbol{u}\right) = rT(\boldsymbol{u})$$\qed
\end{tcolorbox}

\end{document}