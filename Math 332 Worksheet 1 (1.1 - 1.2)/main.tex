\documentclass[letter,11pt]{article}
\usepackage{xcolor}
\usepackage{float}
\usepackage{amsthm}
\usepackage{amssymb}
\usepackage{wrapfig}
\usepackage{tabularx}
\usepackage{titlesec}
\usepackage{tikz}
\usepackage{geometry}
\usepackage{verbatim}
\usepackage{enumitem}
\usepackage{fancyhdr}
\usepackage{multicol}
\usepackage{systeme}
\usepackage{graphicx}
\usepackage{mathtools}
\usepackage{booktabs}
\usepackage{svg}
\usepackage[most]{tcolorbox}
\usepackage[T1]{fontenc}
\usetikzlibrary{trees}
\setlength{\multicolsep}{0pt} 
\pagestyle{fancy}
%\fancyhf{} % clear all header and footer fields
\fancyhead{}\fancyfoot{}
\fancyhead[R]{\textbf{\thepage}}
\fancyhead[L]{Aiden M. Rosenberg, MMXXIV A.D. }
\addtolength{\headwidth}{3cm}
\addtolength{\headheight}{1cm}

\usepackage{pgfplots}
\pgfplotsset{compat=1.17}
\usepgfplotslibrary{fillbetween}

\usepackage{pst-plot}
\usepgfplotslibrary{polar}

\renewcommand{\headrulewidth}{1pt}
\renewcommand{\footrulewidth}{0pt}
\geometry{left=1.5cm, top=2.5cm, right=1.5cm, bottom=2cm}

\usepackage[most]{tcolorbox}

\usepackage{tasks}
\settasks{
	label=(\Alph*.),
	label-width=21pt
}

\raggedright
\setlength{\tabcolsep}{0in}

% Sections formatting
\titleformat{\section}{
  \vspace{-4pt}\scshape\raggedright\large
}{}{0em}{}[\color{black}\titlerule \vspace{-7pt}]

\titleformat{\subsection}[block]
  { \vspace{4pt}\bfseries\centering}
  {}{0em}{}

\newcommand{\pvec}[1]{\vec{#1}\mkern2mu\vphantom{#1}}

\begin{document}

\thispagestyle{empty}

%----------HEADING-----------------

\parbox{2.35cm}{%
	\includesvg[width=2.3cm]{logo.svg}
}
\parbox{0.3cm}{\hspace{0.3cm}}
\parbox{\dimexpr\linewidth-5cm\relax}{
	\setlength{\tabcolsep}{0.5em}
	\def\arraystretch{1.25}
	\begin{tabular}{@{}llll@{}}
		\toprule
		\multicolumn{4}{c}
		{\hspace{-0.5em}\textbf{Assignment}: Worksheet \#1 (\S 1.1–1.2)} \\ \midrule
		\textbf{Name:}   & Aiden M. Rosenberg  & \textbf{Professor:} & Dr. Terry Bridgman Ph.D \\
		\textbf{Course:} & Linear Algebra          & \textbf{Date:}      & \today \: A.D.   \\ \bottomrule
	\end{tabular} }
\vspace{1cm}

\section{Problem 1}
Consider the following matrices,
\begin{tasks}(4)
    \task $\begin{bmatrix}2 & 0 & 0 \\ 0 & 2 & 0 \\ 0 & 0 & 2\end{bmatrix}$
    \task $\begin{bmatrix}1 & 3 & 5 \\ 2 & 3 & 0 \\ 1 & 0 & 0\end{bmatrix}$
    \task $\begin{bmatrix}1 & 0 & 0 & 0 \\ 0 & 1 & 2 & 0 \\ 0 & 0 & 0 & 1\end{bmatrix}$
    \task $\begin{bmatrix}1 & 1 & 0 & 1 \\ 0 & 0 & 1 & 0 \\ 0 & 0 & 0 & 1\end{bmatrix}$
    \task $\begin{bmatrix}1 & 0 & 2 & -1\end{bmatrix}$
    \task $\begin{bmatrix}2 & 0 \\ 0 & 0\end{bmatrix}$
    \task $\begin{bmatrix}1 & 0 & 2 & 3 \\ 0 & 1 & 0 & 1 \\ 0 & 1 & 2 & 0\end{bmatrix}$
    \task $\begin{bmatrix}0 & 0 & 0 & 0 \\ 1 & 0 & 0 & 0 \\ 0 & 1 & 0 & 0 \\ 0 & 0 & 1 & 0\end{bmatrix}$
\end{tasks}

\begin{enumerate}[label = \alph*.]
    \item Which of the matrices given above represent a matrix in row echelon form?
    $$\boxed{\text{A, C, D, E, F}}$$
    \item  Which of the options given above represent a matrix in reduced row echelon form?
    $$\boxed{\text{C, E}}$$
\end{enumerate}

\section{Problem 2} Which of the following options is the reduced row echelon form of the following matrix?

$$\begin{bmatrix} 1 & 2 & -1 & -1 \\ 2 & 4 & -1 & 0 \\ -3 & -6 & 1 & 0 \end{bmatrix}$$

\begin{tasks}(4)
\task $\begin{bmatrix}1 & 0 & 1 & 0 \\ 0 & 1 & 1 & 0 \\ 0 & 0 & 0 & 1\end{bmatrix}$
\task $\begin{bmatrix}1 & 0 & 1 & 0 \\ 0 & 1 & 1 & 0 \\ 0 & 0 & 0 & 1\end{bmatrix}$
\task $\begin{bmatrix}1 & 2 & 0 & 0 \\ 0 & 0 & 1 & 0 \\ 0 & 0 & 0 & 1\end{bmatrix}$
\task $\begin{bmatrix}1 & 2 & 0 & 0 \\ 0 & 0 & 1 & 1 \\ 0 & 0 & 0 & 0\end{bmatrix}$
\end{tasks}

\begin{tcolorbox}[boxrule=1mm, width=(.9\linewidth),before=\hfill,after=\hfill, adjusted title={Problem 2 Solution}]
Matrix (C)
\tcblower
$$\begin{bmatrix}
1 & 2 & -1 & -1 \\
2 & 4 & -1 & 0 \\
-3 & -6 & 1 & 0
\end{bmatrix} \xrightarrow{R_2 = R_2 - 2R_1}
\begin{bmatrix}
1 & 2 & -1 & -1 \\
0 & 0 & -1 & 2 \\
-3 & -6 & 1 & 0
\end{bmatrix}\xrightarrow{R_3 = R_3 +3R_1}
\begin{bmatrix}
1 & 2 & -1 & -1 \\
0 & 0 & 1 & 2 \\
0 & 0 & -2 & -3
\end{bmatrix}\xrightarrow{R_1 = R_1 +R_2}$$

$$\begin{bmatrix}
1 & 2 & 0 & 0 \\
0 & 0 & 1 & 1 \\
0 & 0 & 0 & 0
\end{bmatrix} \xrightarrow{R_2 = R_2 - 2R_1}
\begin{bmatrix}
1 & 2 & -1 & -1 \\
0 & 0 & 1 & 2 \\
0 & 0 & -2 & -3
\end{bmatrix}\xrightarrow{R_3 = R_3 +3R_1}
\begin{bmatrix}
1 & 2 & -1 & -1 \\
0 & 0 & 1 & 2 \\
0 & 0 & -2 & -3
\end{bmatrix}\xrightarrow{R_1 = R_1 +R_2}$$

$$\begin{bmatrix}
1 & 2 & 0 & 1 \\
0 & 0 & 1 & 2 \\
0 & 0 & -2 & -3
\end{bmatrix}\xrightarrow{R_3 = R_3 +2R_2}
\begin{bmatrix}
1 & 2 & 0 & 1 \\
0 & 0 & 1 & 2 \\
0 & 0 & 0 & 1
\end{bmatrix}\xrightarrow{R_1 = R_1 -R_3}
\begin{bmatrix}
1 & 2 & 0 & 0 \\
0 & 0 & 1 & 2 \\
0 & 0 & 0 & 1
\end{bmatrix}\xrightarrow{R_2 = R_2 -2R_3}$$

$$\begin{bmatrix}
1 & 2 & 0 & 0 \\
0 & 0 & 1 & 1 \\
0 & 0 & 0 & 0
\end{bmatrix} \Longrightarrow \text{Matrix (C)}$$

\end{tcolorbox}

\section{Problem 3} Solve the system.

\begin{align*}
x_{1}+5 x_{2} & =-2 \\
-3 x_{2}+x_{3} & =8 \\
2 x_{1}+9 x_{2}+2 x_{3} & =7
\end{align*}

\begin{tcolorbox}[boxrule=1mm, width=(.9\linewidth),before=\hfill,after=\hfill, adjusted title={Problem 3 Solution}]
$$x_1=5,x_2=3,x_3=-1$$
\tcblower

$$\begin{bmatrix}
1 & 5 & 0 & -2 \\
0 & -3 & 1 & 8 \\
2 & 9 & 2 & 7
\end{bmatrix} \xrightarrow{R_3 = R_3 -2R_1}
\begin{bmatrix}
1 & 5 & 0 & -2 \\
0 & -3 & 1 & 8 \\
0 & -1 & 2 & 11
\end{bmatrix} \xrightarrow[R_3=-R_2]{R_2 = R_3}
\begin{bmatrix}
1 & 5 & 0 & -2 \\
0 & 1 & -2 & -11 \\
0 & -3 & 1 & 8
\end{bmatrix}\xrightarrow{R_3 = R_3 +3R_2}$$
$$\begin{bmatrix}
1 & 5 & 0 & -2 \\
0 & 1 & -2 & -11 \\
0 & 0 & -5 & -25
\end{bmatrix}\xrightarrow[R_2=R_2+2R_3]{R_3 = -\frac{1}{5}R_3}
\begin{bmatrix}
1 & 5 & 0 & -2 \\
0 & 1 & 0 & -1 \\
0 & 0 & 1 & 5
\end{bmatrix}\xrightarrow{R_1=R_1-5R_2}
\begin{bmatrix}
1 & 0 & 0 & 3 \\
0 & 1 & 0 & -1 \\
0 & 0 & 1 & 5
\end{bmatrix}$$ 
$$\Longrightarrow  x_{1}=5, x_{2}=3, x_{3}=-1$$
\end{tcolorbox}
\newpage

\section{Problem 4}
Do the planes, $x+2 y+z=4, \quad y-z=1 \quad \text { and } \quad x+3 y=0$ have at least one common point of intersection? Why or why not?
\vspace{1cm}
\begin{tcolorbox}[boxrule=1mm, width=(.9\linewidth),before=\hfill,after=\hfill, adjusted title={Problem 4 Solution}]
The system of given planes does not have a unique point of intersection, as evidenced by its inconsistency in echelon form, specifically $0\cdot z \neq -5$ as seen below.
\tcblower

$$\begin{bmatrix}
1 & 2 & 1 & 4 \\
0 & 1 & -1 & 1 \\
1 & 3 & 0 & 0
\end{bmatrix}\xrightarrow{R_3=R_3-R_1}
\begin{bmatrix}
1 & 2 & 1 & 4 \\
0 & 1 & -1 & 1 \\
0 & 1 & -1 & -4
\end{bmatrix}\xrightarrow{R_3=R_3+R_2}
\begin{bmatrix}
1 & 2 & 1 & 4 \\
0 & 1 & -1 & 1 \\
0 & 0 & 0 & -5
\end{bmatrix}$$
\end{tcolorbox}

\section{Problem 5} Using row reduction, solve the system,

\begin{align*}
-x_{2}-x_{3}+x_{4} & =0 \\
x_{1}+x_{2}+x_{3}+x_{4} & =6 \\
2 x_{1}+4 x_{2}+x_{3}-2 x_{4} & =-1 \\
3 x_{1}+x_{2}-2 x_{3}+2 x_{4} & =3
\end{align*}

\begin{tcolorbox}[boxrule=1mm, width=(.9\linewidth),before=\hfill,after=\hfill, adjusted title={Problem 5 Solution}]
$$x_{1}=2, x_{2}=-1, x_{3}=3, x_{4}=2$$
\tcblower

$$\begin{bmatrix}
0 & -1 & -1 & 1 & 0 \\
1 & 1 & 1 & 1 & 6 \\
2 & 4 & 1 & -2 & -1 \\
3 & 1 & -2 & 2 & 3
\end{bmatrix}\xrightarrow[R_2=R_1]{R_1=R_2}
\begin{bmatrix}
1 & 1 & 1 & 1 & 6 \\
0 & -1 & -1 & 1 & 0 \\
2 & 4 & 1 & -2 & -1 \\
3 & 1 & -2 & 2 & 3
\end{bmatrix}\xrightarrow[R_4=R_3-3R_1]{R_3=R_3-2R_1}$$
$$\begin{bmatrix}
1 & 1 & 1 & 1 & 6 \\
0 & -1 & -1 & 1 & 0 \\
0 & 2 & -1 & -4 & -13 \\
0 & -2 & -5 & -1 & -15
\end{bmatrix}\xrightarrow[R_4=R_4-2R_2]{R_3=2R_2+R_3}
\begin{bmatrix}1 & 1 & 1 & 1 & 6 \\
0 & 1 & 1 & -1 & 0 \\
0 & 0 & -3 & -2 & -13 \\
0 & 0 & -3 & -3 & -15
\end{bmatrix}\xrightarrow{R_4=R_4-R_3}$$
$$\begin{bmatrix}
1 & 1 & 1 & 1 & 6 \\
0 & 1 & 1 & -1 & 0 \\
0 & 0 & -3 & -2 & -13 \\
0 & 0 & 0 & 1 & 2
\end{bmatrix}\xrightarrow{R_3=-\frac{1}{3}\left(R_3+2R_4\right)}
\begin{bmatrix}
1 & 1 & 1 & 1 & 6 \\
0 & 1 & 1 & -1 & 0 \\
0 & 0 & 1 & 0 & 3 \\
0 & 0 & 0 & 1 & 2
\end{bmatrix} \xrightarrow[R_1=R_1-\left(R_2+R_3+R_4\right)]{R_2=R_2+R_4-R_3}$$
$$\begin{bmatrix}
1 & 0 & 0 & 0 & 2 \\
0 & 1 & 0 & 0 & -1 \\
0 & 0 & 1 & 0 & 3 \\
0 & 0 & 0 & 1 & 2
\end{bmatrix} \Longrightarrow x_{1}=2, x_{2}=-1, x_{3}=3, x_{4}=2
$$

\end{tcolorbox}
\section{Problem 6}
Consider a linear system whose augmented matrix is of the form

$$\begin{bmatrix} 1 & 1 & 3 & 2\\ 1 & 2 & 4 & 3\\ 1 & 3 & a & b \end{bmatrix}$$
\begin{enumerate}[label = \alph*.]
    \item For what values of $a$ and $b$ will the system have infinitely many solutions?
    \item For what values of $a$ and $b$ will the system be inconsistent?
    \item For what values of $a$ and $b$ will the system have a unique solution?
\end{enumerate}

\begin{tcolorbox}[boxrule=1mm, width=(.9\linewidth),before=\hfill,after=\hfill, adjusted title={Problem 6 Solution}]
$$\begin{bmatrix}
1 & 1 & 3 & 2 \\
1 & 2 & 4 & 3 \\
1 & 3 & a & b
\end{bmatrix} \xrightarrow[R_3=R_3-R_1]{R_2=R_2-R_1}
\begin{bmatrix}
1 & 1 & 3 & 2 \\
0 & 1 & 1 & 1 \\
0 & 2 & a-3 & b-2
\end{bmatrix}\xrightarrow{R_3=R_3-2R_2}
\begin{bmatrix}
1 & 1 & 3 & 2 \\
0 & 1 & 1 & 1 \\
0 & 0 & a-5 & b-4
\end{bmatrix}
$$
\tcblower
\begin{enumerate}[label = \alph*.]
    \item The system will be consistent with an infinite number of solutions when $a=5$ and $b=4$  since $a-5 = 0$ and $b-4 = 0$ is true as shown in the echelon form. 
    \item The system will be inconsistent when $a=5$ and $b\neq4$  since $a-5 = 0$ and $b-4 \neq 0$ is true as shown in the echelon form. 
    \item The system will have a unique solution when $a\neq5$, $b$ can be any number since $ax_n = b$. 
\end{enumerate}
\end{tcolorbox}

\end{document}